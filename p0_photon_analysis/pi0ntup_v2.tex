\documentclass[pdftex,12pt,letter]{article}
\usepackage[margin=0.75in]{geometry}
\usepackage{verbatim}
\usepackage{graphicx}
\usepackage{xspace}
\usepackage{cite}
\usepackage{url}
\usepackage[pdftex,pdfpagelabels,bookmarks,hyperindex,hyperfigures]{hyperref}

\newcommand{\sqsn}{\mbox{$\sqrt{s_{_{NN}}}$}\xspace}
\newcommand{\piz}{\mbox{$\pi^0$}\xspace}
\newcommand{\pt}{\mbox{$p_{\rm T}$}\xspace}
\newcommand{\gev}{\mbox{GeV}\xspace}
\newcommand{\gevc}{\mbox{GeV/$c$}\xspace}

\title{Neutral meson and photon ntuples derived from PHENIX Data: explanation and examples}
\date{\today}
\author{G.\,David\\
{ \it Stony Brook University, Brookhaven National Laboratory}}


\begin{document}
\maketitle

\begin{abstract}
\noindent  This writeup contains information about small samples of reconstructed $p$+Au
data taken with the PHENIX detector in 2015 at \sqsn=200\,GeV. These samples were created
for educational and training purposes only and are not at useable for deriving any physics
results. A few analysis examples are provided with a focus on the Electromagnetic Calorimeter (EMCAL),
as well as methods of identifying $\pi^0$-s and photons using that device.
\end{abstract}

%%%%%%%%%%%%%%%%%%%%%%%%%%%%%%%%%%%%%%%%%%%%%%%%%%%%%

\section{Introduction}
Data samples described in this paper contain
information on {\it clusters} -- small, contiguous regions of energy
deposit in the EMCal -- which are usually considered as 
{\it energy deposited by one single particle}.  While this is not
always true, this is our first working hypothesis, but we should
always be aware that a cluster might contain energy from more than one
particle.  Also, not all
clusters correspond to photons; many are from hadrons, especially at
low and very high \pt (transverse momentum)\footnote{This may sound
  surprising in an electromagnetic calorimeter, but it is true: many
  high \pt single clusters are actually from two overlapping photons
  from a \piz decay, i.e. they are hadrons.
}.  
In fact, it's going to
be your job to find selection criteria (``cuts'') which eliminate
hadrons (the {\it contamination}) but still preserves most of the
photons ({\it signal}) in the sample.

As for \piz, recall that most of them decay via 
$\pi^0 \rightarrow \gamma\gamma$, so they can be reconstructed from
photon candidate {\it pairs} using the invariant mass
$$ m_{\gamma\gamma} = \sqrt{2E_1E_2(1-cos(\theta))} $$
where $E_1,E_2$ are the energies of the two photon candidates and
$\theta$ is the {\it opening angle} between them.  Note that the \pt
of the \piz is simply the vector sum of the \pt-s of the two photons
paired. 

Please note that -- as per PHENIX standards -- time is always given in
nanoseconds, distance in centimeters and energy in GeV.


\section{The Data:  {\it gnt} and {\it ggntuple} Ntuples}
The two files {\it MBntup.root} and {\it ERTntup.root} are produced
using a small fraction of the \sqsn=200\,GeV $p$+Au data collected by
PHENIX in 2015.  They are published for educational purposes only.
You will not be able to derive any publishable physics results from
them, however, you can learn some basic techniques of
particle identification (PID), practice how to make different cuts and
estimate their effect, how the signal to background ratio can be
improved at the expense of some loss in the signal and how to find an
optimum between the two trends.  You may also learn that PID itself is
also not an unambiguous procedure: you can apply much looser cuts on
the ``photonness'' of the clusters when reconstructing \piz, because
you identify the \piz by the {\it correlation} of the two ``photons'',
namely, whether their invariant mass is in the expected range (around
0.135\,\gev) or not.  Only photon pairs from the decay of the same
\piz are truly correlated, random pairs of photons, photon-hadron or
hadron-hadron pairs only rarely give invariant masses in the 
``\piz window''.  Therefore, you can allow yourself to have more
contamination is the ``photon'' sample -- and gain significant
statistics in reconstructed \piz (why?).  --  On the other hand for
reconstructing (inclusive) photons you don't have such a help: if you
decide that a cluster is a photon, that's it.  In order to keep the
contamination from misclassified hadrons you might want to make your
photon PID cuts stricter, in order to increase {\it purity} 
(in other terminology decrease {\it contamination}) at the
expense of {\it efficiency}.

The first Ntuple is called {\it ggntuple} (gamma-gamma ntuple) and
contains information on EMCal cluster {\it pairs} and the two clusters
in the pair (see Table~\ref{tab:ggntuple}).  In each event all
clusters in a sector are paired with all other clusters, and the pair
variables ({\it pt, costheta, phi, mass, asym}) calculated, even if
they are obviously not photons from a \piz decay (for instance because
their invariant mass is, say, 0.4\,\gev).  These ``obviously wrong''
random pairs are called the {\it combinatorial background} and they
help you to estimate the background in the \piz peak when you'll try
to extract the number of \piz in the sample.
The momenta attributed to a cluster (particle candidate) is calculated
from the vector connecting the collision point ({\it vtxZ}) with the
impact point on the EMCal, and under the assumption that the particle
is a photon (i.e. $p=E$, the measured energy).  From these, the pair
momentum {\it pt} is calculated as the vector sum of the transverse
components; {\it costheta} is the cosine of the opening angle between
the two momentum vectors, {\it phi} is the azimuth of the pair
momentum (i.e. of the parent \piz), {\it mass} is the invariant mass
calculated with the formula above, and {\it asym} is the energy
asymmetry $\alpha$ of the two clusters
$$\alpha = \frac{|E_1-E_2|}{E_1+E_2}$$
The rest of the variables describe the two clusters ({\it 1,2}), their
meaning is spelled out below.

\begin{table}[h]
  \begin{tabular}{|l|l|} \hline
    Variable name & Description \\ \hline
    cent & Event centrality \\
    vtxZ & $z$-vertex of the event \\
    pt   & Transverse momentum of cluster {\it pair} ($\pi^0$
    candidate) \\
    costheta & $cos$ of the opening angle between the two clusters \\    
    phi & Azimuthal angle of the direction of the pair momentum
    (assumed $\pi^0$) \\
    mass & Invariant mass calculated from the two clusters (energy and
    position) \\
    asym & Energy asymmetry ($|E_1-E_2|/(E_1+E_2)$) of the two clusters \\
    sec1 & EMCal sector where the first cluster is \\
    Ecore1 & ``Core'' energy of the first cluster \\
    tof1   & Time-of-flight of the first cluster \\
    twrhit1 & Number of towers in the first cluster \\
    prob1 & Probability that the first cluster is a photon (based on
    $\chi^2$) \\
    chisq1 & $\chi^2$ from expected photon shape for the first cluster
    \\
    stoch1 & Combined variable to describe ``photonness'' of first
    cluster \\
    sec2 & ...  Same quantities as above for the second cluster of the
    pair \\ \hline
  \end{tabular}
  \vspace{0.3cm}
  \caption{Fields in the $\gamma$-pair Ntuple ({\it ggntuple}).
}
  \label{tab:ggntuple}
 \end{table}


The second Ntuple is called {\it gnt} and contains information on
individual clusters (again, many of which are {\it not} photons!).
Here {\it pt} is the transverse momentum of the (single) particle,
calculated under the assumption that it is a photon (straight path
from the collision vertex and $p=E$), {\it costheta} is the cosine of
the polar angle, {\it phi} is the azimuth, and {\it sec} is the sector
number (0-5 are lead scintillator sectors, PbSc, 6-7 are lead glass,
PbGl).  For both detectors the smallest, individually read out units
are called {\it towers}.

The field {\it ecore} is an estimate of the particle energy (using various
corrections to account for detector effects), {\it ecent} is the
energy in the central (highest energy) tower of the cluster, 
{\it tof} is the time-of-flight measured in the central tower and
corrected with the flight-path $s/c$ such that for photons {\it tof}
is a distribution around zero.  Based on testbeam data we have a model
(parametrization) of electromagnetic showers depending on energy, impact
point and angle of the photon; the {\it chisq} is calculated as the
deviation of the actual shower (energy deposit pattern in the towers)
from the ``model'' shower of the same energy, impact point and angle.
The {\it prob} is calculated from this $\chi^2$, usually showers with
$prob>0.05$ are very likely indeed photons;  {\it disp} is the 2D
dispersion, {\it twrhit} is the number of towers hit (i.e. included in
the cluster), finally {\it stoch} is a combination of various
probability measures to characterize ``photonness'', higher values
mean higher probability that the cluster is indeed a photon.  The
calculated impact point of the particle is given as $x,y,z$ in the
PHENIX absolute coordinate system ($x>0$ is the West Arm, negative $z$
is South).

The {\it MBntup.root} file is produced from minimum bias data (although
with a lower limit  \pt$>2.0$\,\gevc in {\it gnt} and pair \pt in
{\it ggntuple}), whereas in {\it ERTntup.root} the threshold for 
single cluster \pt in {\it gnt} is 8\,\gev, and the threshold for pair
\pt in {\it ggntuple} is also 8\,\gev.  Note that since here we
restrict only the pair \pt, the energy of the individual clusters can
be (and often is) significantly lower.  Important: both the MB and the
ERT {\it ggntuple}-s are written out with the condition that the
asymmetry is less than 0.8.

\begin{table}[h]
  \begin{tabular}{|l|l|} \hline
    Variable name & Description \\ \hline
    cent & Event centrality \\
    vtxZ & $z$-vertex of the event \\
    pt   & Transverse momentum of the cluster ($\gamma$-candidate) \\
    costheta & Polar angle of the cluster ($\gamma$-candidate) \\    
    phi & Azimuthal angle of the cluster ($\gamma$-candidate) \\
    sec & EMCal sector of the cluster ($\gamma$-candidate) \\
    ecore & ``Core'' energy of the cluster ($\gamma$-candidate) \\
    ecent & Energy in the central tower of the cluster ($\gamma$-candidate) \\
    tof   & Time-of-flight in the central tower of the cluster ($\gamma$-candidate) \\
    prob & Probability that the cluster is a photon (based on $\chi^2$) \\
    disp & Dispersion of the cluster ($\gamma$-candidate) \\
    chisq & $\chi^2$ from expected photon shape of the cluster ($\gamma$-candidate) \\
    twrhit & Number of towers in the cluster ($\gamma$-candidate) \\
    stoch & Combined variable to describe ``photonness'' of the cluster ($\gamma$-candidate) \\
    x & $x$-position of impact point on the EMCal surface \\
    y & $y$-position of impact point on the EMCal surface \\
    z & $z$-position of impact point on the EMCal surface \\ \hline
  \end{tabular}
  \vspace{0.3cm}
  \caption{Fields in the $\gamma$ Ntuple ({\it gnt}).
}
  \label{tab:gnt}
 \end{table}
\section{TBD...}




\end{document}



